\section{Einleitung}\label{sec:einleitung}
Im Zuge der fortschreitenden Digitalisierung und Automatisierung gewinnen mobile Robotersysteme zunehmend an Bedeutung. 
Insbesondere im Bereich der autonomen Navigation und Echtzeit-Umgebungswahrnehmung eröffnen sich durch den Einsatz moderner Sensortechnik und künstlicher Intelligenz neue Anwendungsmöglichkeiten. 
Diese Entwicklungen stellen jedoch nicht nur technische Herausforderungen dar, sondern erfordern auch ein fundiertes Verständnis für die zugrunde liegenden Technologien und deren Zusammenspiel.
\newPar
Die vorliegende Projektarbeit widmet sich dem Aufbau und der praktischen Umsetzung eines solchen Robotersystems auf Basis des TurtleBot3. 
Im Fokus steht dabei die Integration eines auf Deep Learning basierenden Objekterkennungsmodells (YOLO) in ein mit dem Robot Operating System 2 (ROS 2) gesteuertes System. 
Die Kombination dieser beiden Technologien erlaubt die Entwicklung eines autonomen Roboters, der visuelle Reize in seiner Umgebung interpretieren und darauf in Form von gezielten Bewegungen reagieren kann.
\newPar
Die Arbeit verfolgt damit einen Ansatz, der Kenntnisse aus den Bereichen Robotik, Softwareentwicklung, Bildverarbeitung und künstliche Intelligenz zusammenführt. 
Durch den starken Praxisbezug soll zum einen ein funktionales System entstehen, als auch ein tieferes Verständnis für die Herausforderungen und Möglichkeiten moderner mobiler Robotik erlangt werden.
\subsection{Motivation}
Die Motivation für die vorliegende Arbeit ist es, theoretisches Wissen mit praktischer Erfahrung zu verknüpfen und einen Einblick in die komplexe Welt mobiler Robotiksysteme zu gewinnen.
Ein zentrales Lernziel des Projekts ist die Einarbeitung in das Robot Operating System 2 (ROS 2), das für viele moderne Robotik-Anwendungen als technologische Basis dient. 
Da die Projektgruppe bislang keine praktischen Erfahrungen mit ROS 2 gesammelt hat, ist eine gründliche Auseinandersetzung mit dessen Architektur, Kommunikationsprinzipien und Konfigurationsmöglichkeiten erforderlich. 
Im Vergleich zu anderen Entwicklungsumgebungen gilt ROS 2 als besonders leistungsfähig, bringt jedoch auch eine hohe Komplexität mit sich, die gerade für Einsteiger eine steile Lernkurve bedeutet. 
Bereits die Installation, Inbetriebnahme und das Verständnis der systeminternen Abläufe stellen eine Herausforderung dar.
Diese Einarbeitungsphase bildet die technische Grundlage für die spätere Steuerung des TurtleBot3 und ist somit ein wesentlicher Bestandteil der Arbeit.
\newPar
Darüber hinaus möchte sich die Projektgruppe intensiv mit Methoden der Bildverarbeitung und dem Einsatz künstlicher Intelligenz in der Robotik beschäftigen. 
Ziel ist es, ein fundiertes Verständnis für das YOLO-Framework zur Objekterkennung zu entwickeln und dessen praktische Anwendung im Zusammenspiel mit ROS 2 zu erproben. 
Um gezielt Wissen im Bereich des Trainings von KI-Modellen zu erwerben, soll zudem ein bestehendes YOLO-Modell weitertrainiert und auf spezifische Erkennungsobjekte angepasst werden.
Die Kombination beider Technologien stellt eine anspruchsvolle, aber zugleich äußerst lehrreiche Herausforderung dar, die es ermöglicht, ein praxisnahes Projekt umzusetzen.
\section{Konzept}\label{sec:konzept}
Ziel dieser Arbeit ist die Entwicklung eines Systems zur Echtzeit-Bilderkennung und -Steuerung eines TurtleBot3-Roboters unter Verwendung eines selbst trainierten YOLO-Modells. 
Dabei soll ein praxisnaher Einblick in das Training und den Einsatz künstlicher Intelligenz im Bereich der mobilen Robotik gewonnen werden. 
Das trainierte Modell soll aufgrund der erhöhten Rechenanforderungen auf einem separaten Rechner mit GPU ausgeführt werden, während der Roboter über ROS 2 Kamera- und Steuerdaten mit dem Rechner austauscht. 
Zur Demonstration der Funktionalität sollen Richtungspfeile als Erkennungsobjekte verwendet werden. 
Erkennt das System einen Pfeil, soll der Roboter entsprechend abbiegen.
Dieses Szenario ermöglicht einen gezielten Einblick in die Objekterkennung und die Verwendung von ROS.
\newPar
Speziell zur Anwendung kommen soll das Bildverarbeitungsframework YOLO, das mit einer Vielzahl von vortrainierten Objektklassen bereits eine gute Basis zur Objekterkennung bietet.
Dieses kann mittels Tools des Herstellers Ultralytics bzw. der Open\hyp{}Source\hyp{}Bibliothek PyTorch speziell nachtrainiert werden, um eigene benutzerdefinierte Objekte erkennen zu können.